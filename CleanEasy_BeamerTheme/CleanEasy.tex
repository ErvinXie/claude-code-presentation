%----------------------------------------------------------------------------------------
%    PACKAGES AND THEMES
%----------------------------------------------------------------------------------------
\documentclass[aspectratio=169,xcolor=dvipsnames]{beamer}
\makeatletter
\def\input@path{{theme/}}
\makeatother
\usetheme{CleanEasy}
\usepackage[utf8]{inputenc}
\usepackage{lmodern}
\usepackage[T1]{fontenc}
\usepackage{fix-cm}
\usepackage{amsmath}
\usepackage{mathtools}
\usepackage{listings}
\usepackage{xcolor}
\usepackage{hyperref}
\usepackage{graphicx}
\usepackage{booktabs}
\usepackage{tikz}
\usetikzlibrary{positioning, shapes, arrows, calc, decorations.pathreplacing, arrows.meta, backgrounds, patterns, overlay-beamer-styles}
\usepackage{etoolbox}
\usepackage{xeCJK}
\setCJKmainfont{STSong}
\setCJKsansfont{STHeiti}
\setCJKmonofont{STFangsong}

%----------------------------------------------------------------------------------------
%    LAYOUT CONFIGURATION
%----------------------------------------------------------------------------------------
\input{configs/configs}
%----------------------------------------------------------------------------------------
%    TITLE PAGE
%----------------------------------------------------------------------------------------


%---------------------------------------------


\title[Claude Code 使用分享]{如何用 Claude Code 提升开发效率}

\author{谢尔文}

\institute{趋动科技}


\vspace{-2cm}\date{2025 年 10 月}
% Define positions for logos on title page
\titlegraphic{
  \begin{tikzpicture}[remember picture, overlay]
    % 
    \node[anchor=south west, xshift=0.5cm, yshift=0cm] at (current page.south west) {
      \includegraphics[height=1.3cm]{logos/CleanEasy-logo1.png}
    };

    % 
    \node[anchor=south west, xshift=2.5cm, yshift=0cm] at (current page.south west) {
      \includegraphics[height=0.9cm]{logos/CleanEasy-logo2.png}
    };
    
    % 
    \node[anchor=north east, xshift=-0.8cm, yshift=-0.3cm] at (current page.north east) {
      \includegraphics[height=1.1cm]{logos/CleanEasy-logo3.png}
    };
    
    %
    \node[anchor=north east, xshift=-2.8cm, yshift=-0.3cm] at (current page.north east) {
      \includegraphics[height=1.4cm]{logos/CleanEasy-logo4.png}
    };
  \end{tikzpicture}
}


%----------------------------------------------------------------------------------------


\begin{document}

\begin{frame}[plain]
  \titlepage
\end{frame}

\begin{frame}[plain]{目录}
  \tableofcontents
\end{frame}

%----------------------------------------------------------------------------------------
\section{开场白}
%----------------------------------------------------------------------------------------

\begin{frame}{我的 Claude Code 使用体验}
  \begin{block}{今年最幸运的一件事}
    用上 Claude Code 是我今年最幸运的事。它给我的震撼不亚于、甚至超过了当初初次使用 ChatGPT 3.5 时的感受。
  \end{block}

  \vspace{0.5cm}

  \begin{columns}
    \begin{column}{0.48\textwidth}
      \textbf{使用时间:}从 8 月至今约 2 个月

      \vspace{0.3cm}

      \textbf{完成的项目:}
      \begin{itemize}
        \item 系统级代码开发
        \item 运维自动化
        \item 各类小工具
        \item 项目管理工作
      \end{itemize}
    \end{column}
    \begin{column}{0.48\textwidth}
      \begin{alertblock}{核心价值}
        让我有能力实现各种想法,效率提升 5-10 倍
      \end{alertblock}
    \end{column}
  \end{columns}
\end{frame}

%----------------------------------------------------------------------------------------
\section{基础概念:什么是终端}
%----------------------------------------------------------------------------------------

\begin{frame}{终端是什么?}
  \begin{columns}
    \begin{column}{0.48\textwidth}
      \textbf{通俗理解}
      \begin{itemize}
        \item 就像手机上的"文件管理器"
        \item 用文字命令控制电脑
        \item 程序员的"工作台"
        \item 比图形界面更强大、更快
      \end{itemize}

      \vspace{0.3cm}

      \textbf{日常类比}
      \begin{itemize}
        \item 图形界面 = 用遥控器看电视
        \item 终端 = 直接对电视说话
      \end{itemize}
    \end{column}
    \begin{column}{0.48\textwidth}
      \textbf{终端能做什么}
      \begin{itemize}
        \item 创建、移动、删除文件
        \item 运行程序
        \item 查看系统信息
        \item 批量处理任务
      \end{itemize}

      \vspace{0.3cm}

      \begin{exampleblock}{示例}
        \small
        \texttt{cd project} - 进入项目文件夹\\
        \texttt{ls} - 列出所有文件\\
        \texttt{python app.py} - 运行程序
      \end{exampleblock}
    \end{column}
  \end{columns}
\end{frame}

\begin{frame}{为什么 Claude Code 基于终端?}
  \begin{enumerate}
    \item \textbf{信息密度高}
    \begin{itemize}
      \item 终端能看到文件结构、代码内容、运行结果
      \item 一个屏幕包含海量信息
      \item AI 可以快速理解整个项目状态
    \end{itemize}

    \vspace{0.3cm}

    \item \textbf{操作能力强}
    \begin{itemize}
      \item 可以直接修改文件、运行代码
      \item 不需要鼠标点来点去
      \item 一条命令完成复杂操作
    \end{itemize}

    \vspace{0.3cm}

    \item \textbf{程序员的自然环境}
    \begin{itemize}
      \item 程序员本来就在终端工作
      \item AI 和人在同一个"工作台"上协作
      \item 真正站在同一信息起跑线
    \end{itemize}
  \end{enumerate}
\end{frame}

%----------------------------------------------------------------------------------------
\section{Claude Code 是什么}
%----------------------------------------------------------------------------------------

\begin{frame}{Claude Code 定位}
  \begin{columns}
    \begin{column}{0.48\textwidth}
      \textbf{核心功能}
      \begin{itemize}
        \item AI 驱动的编程助手
        \item 命令行界面操作
        \item 深度集成文件系统
        \item 自主执行任务
      \end{itemize}
    \end{column}
    \begin{column}{0.48\textwidth}
      \textbf{和其他工具的区别}
      \begin{itemize}
        \item vs IDE:理解更智能
        \item vs Copilot:能完成整个任务
        \item vs ChatGPT:直接操作代码
      \end{itemize}
    \end{column}
  \end{columns}

  \vspace{0.5cm}

  \begin{block}{最大优势}
    基于命令行和文件系统,能拿到海量的上下文信息
  \end{block}
\end{frame}

\begin{frame}{核心优势}
  \begin{columns}
    \begin{column}{0.48\textwidth}
      \textbf{最优协作模式}
      \begin{itemize}
        \item AI:处理海量背景信息
        \item 人类:提供反馈和创意
        \item 基于文件系统,上下文获取能力极强
      \end{itemize}

      \vspace{0.5cm}

      \textbf{真正能干活}
      \begin{itemize}
        \item 不只是提建议,直接动手
        \item 自动拆解多步骤任务
      \end{itemize}
    \end{column}
    \begin{column}{0.48\textwidth}
      \textbf{工具集成}
      \begin{itemize}
        \item 编辑器、终端
        \item Git、包管理器
        \item 构建、测试工具
        \item 无缝切换
      \end{itemize}

      \vspace{0.5cm}

      \begin{exampleblock}{本质}
        AI 和人站在同一信息起跑线,充分发挥 AI 实力
      \end{exampleblock}
    \end{column}
  \end{columns}
\end{frame}

% \begin{frame}{输入方式的优化空间}
%   \begin{columns}
%     \begin{column}{0.48\textwidth}
%       \textbf{当前瓶颈}
%       \begin{itemize}
%         \item 工位上主要靠打字输入
%         \item 打字速度有限
%         \item 思维 → 文字转换慢
%       \end{itemize}
%
%       \vspace{0.5cm}
%
%       \begin{alertblock}{洞察}
%         输入方式成为人机协作的效率瓶颈
%       \end{alertblock}
%     \end{column}
%     \begin{column}{0.48\textwidth}
%       \textbf{语音交互的潜力}
%       \begin{itemize}
%         \item 语音速度远超打字
%         \item 更自然的表达方式
%         \item 解放双手继续操作
%       \end{itemize}
%
%       \vspace{0.5cm}
%
%       \begin{exampleblock}{未来方向}
%         语音 + Claude Code = 更高效的协作方式
%       \end{exampleblock}
%     \end{column}
%   \end{columns}
%
%   \vspace{0.3cm}
%
%   \small
%   \textit{注:需要考虑工作环境的噪音和隐私问题}
% \end{frame}

%----------------------------------------------------------------------------------------
\section{核心能力概览}
%----------------------------------------------------------------------------------------

\begin{frame}{Claude Code 能做什么}
  \begin{columns}
    \begin{column}{0.48\textwidth}
      \textbf{代码开发}
      \begin{itemize}
        \item 理解项目结构和代码
        \item 智能补全和重构
        \item 诊断和修复 bug
        \item 性能优化
      \end{itemize}

      \vspace{0.3cm}

      \textbf{项目管理}
      \begin{itemize}
        \item Git 操作(提交、分支、PR)
        \item 依赖和构建管理
        \item 自动化测试
      \end{itemize}
    \end{column}
    \begin{column}{0.48\textwidth}
      \textbf{任务协作}
      \begin{itemize}
        \item 自动拆解任务
        \item 追踪执行进度
        \item 并行处理优化
      \end{itemize}

      \vspace{0.3cm}

      \begin{exampleblock}{重点}
        具体能力通过真实案例来展示更直观
      \end{exampleblock}
    \end{column}
  \end{columns}
\end{frame}

%----------------------------------------------------------------------------------------
\section{真实案例}
%----------------------------------------------------------------------------------------

\begin{frame}{案例 1:代码分析 - AWQ 量化格式研究}
  \begin{block}{任务}
    在 sglang 代码库里找到 AWQ 量化的 swizzle 格式排列方式
  \end{block}

  \vspace{0.3cm}

  \begin{columns}
    \begin{column}{0.48\textwidth}
      \textbf{挑战}
      \begin{itemize}
        \item 陌生代码库
        \item 非标准排布方式
        \item 论文描述不清
      \end{itemize}
    \end{column}
    \begin{column}{0.48\textwidth}
      \textbf{效率}
      \begin{itemize}
        \item \textcolor{green}{Claude Code:几分钟}
        \item \textcolor{red}{传统方式:几小时}
      \end{itemize}
    \end{column}
  \end{columns}

  \vspace{0.3cm}

  \begin{exampleblock}{价值}
    快速理解陌生代码,精准定位关键实现
  \end{exampleblock}
\end{frame}

\begin{frame}{案例 2:系统级开发 - KTransformers 支持 AWQ}
  \begin{columns}
    \begin{column}{0.48\textwidth}
      \textbf{任务}
      \begin{itemize}
        \item 实现 group 量化
        \item 高维矩阵切分
        \item AVX 指令优化
      \end{itemize}

      \vspace{0.3cm}

      \textbf{协作}
      \begin{itemize}
        \item AI:写测试、生成框架
        \item 人:调试关键指令
      \end{itemize}
    \end{column}
    \begin{column}{0.48\textwidth}
      \begin{alertblock}{效率提升}
        \textcolor{red}{2 周} → \textcolor{green}{2 天}
      \end{alertblock}

      \vspace{0.5cm}

      \begin{exampleblock}{启示}
        复杂系统级开发也能高效完成
      \end{exampleblock}
    \end{column}
  \end{columns}
\end{frame}

\begin{frame}{案例 3:运维 - GPU 监控中心部署}
  \begin{columns}
    \begin{column}{0.48\textwidth}
      \textbf{任务}
      \begin{itemize}
        \item 多 GPU 集群管理
        \item DCGM + Grafana
        \item Docker 容器化
        \item 批量远程部署
      \end{itemize}
    \end{column}
    \begin{column}{0.48\textwidth}
      \begin{exampleblock}{效率}
        \textcolor{green}{4 小时} vs \textcolor{red}{2-3 天}
      \end{exampleblock}

      \vspace{0.3cm}

      \textbf{价值}
      \begin{itemize}
        \item 运维自动化
        \item 快速部署
      \end{itemize}
    \end{column}
  \end{columns}
\end{frame}

\begin{frame}{案例 4:文档工程 - 立项文档自动化}
  \begin{block}{场景}
    KTransformers 与华为合作立项文档编写
  \end{block}

  \vspace{0.3cm}

  \begin{columns}
    \begin{column}{0.48\textwidth}
      \textbf{传统方式的痛点}
      \begin{itemize}
        \item 疯狂复制粘贴
        \item 反复调格式
        \item 信息整合麻烦
        \item 版本同步困难
      \end{itemize}
    \end{column}
    \begin{column}{0.48\textwidth}
      \textbf{Claude Code 工作流}
      \begin{itemize}
        \item 背景资料 + 模板 → MD
        \item AI 生成初稿
        \item 在 MD 里调整
        \item 复制到 Word
      \end{itemize}
    \end{column}
  \end{columns}

  \vspace{0.3cm}

  \begin{exampleblock}{效果}
    文档编写又快又轻松,专注内容而不是格式
  \end{exampleblock}
\end{frame}

%----------------------------------------------------------------------------------------
\section{最佳实践}
%----------------------------------------------------------------------------------------

\begin{frame}{怎么问问题}
  \begin{enumerate}
    \item \textbf{清楚地描述}
    \begin{itemize}
      \item 说清楚想要什么结果
      \item 描述现在是什么状态
      \item 贴上错误信息
    \end{itemize}

    \vspace{0.3cm}

    \item \textbf{给足上下文}
    \begin{itemize}
      \item 指出相关文件
      \item 说明用的什么技术栈
      \item 讲清楚有什么限制
    \end{itemize}

    \vspace{0.3cm}

    \item \textbf{一步步来}
    \begin{itemize}
      \item 把复杂任务拆开
      \item 每步都验证结果
      \item 及时给反馈
    \end{itemize}
  \end{enumerate}
\end{frame}

\begin{frame}{优化工作流}
  \begin{columns}
    \begin{column}{0.3\textwidth}
      \textbf{工具集成}
      \begin{itemize}
        \item Git
        \item IDE
        \item 终端
        \item CI/CD
      \end{itemize}
    \end{column}
    \begin{column}{0.3\textwidth}
      \textbf{团队协作}
      \begin{itemize}
        \item 代码规范
        \item 文档同步
        \item 知识共享
        \item 最佳实践
      \end{itemize}
    \end{column}
    \begin{column}{0.3\textwidth}
      \textbf{提效技巧}
      \begin{itemize}
        \item 快捷命令
        \item 用模板
        \item 批量操作
        \item 自动化脚本
      \end{itemize}
    \end{column}
  \end{columns}
\end{frame}

%----------------------------------------------------------------------------------------
\section{挑战与解决方案}
%----------------------------------------------------------------------------------------

\begin{frame}{上下文限制问题}
  \begin{block}{现状}
    Claude Code 的上下文一般在几百 K 级别,开发大项目时可能一个特性就用完
  \end{block}

  \vspace{0.3cm}

  \begin{columns}
    \begin{column}{0.48\textwidth}
      \textbf{解决方案}
      \begin{itemize}
        \item 让 AI 写文档记录状态
        \item 定期重构项目结构
        \item 模块化开发
        \item 分阶段处理任务
      \end{itemize}
    \end{column}
    \begin{column}{0.48\textwidth}
      \textbf{为什么不能简单扩展}
      \begin{itemize}
        \item \textcolor{red}{单模型无限长上下文不合理}
        \item 即使能实现,成本也极高
        \item 多 agent + 总结 token 是更优解
        \item 信息检索效率也会降低
      \end{itemize}
    \end{column}
  \end{columns}

  \vspace{0.3cm}

  \begin{alertblock}{未来趋势}
    多 agent 系统会成为主流(Haiku 4.5 已采用)
  \end{alertblock}
\end{frame}

%----------------------------------------------------------------------------------------
\section{核心思考}
%----------------------------------------------------------------------------------------

\begin{frame}{Claude Code 的本质}
  \begin{block}{信息处理加速器}
    任何需要快速处理和输出信息的工作都可以交给它
  \end{block}

  \vspace{0.3cm}

  \begin{exampleblock}{优化流程的思维}
    人负责思考整个流程哪里可以优化
  \end{exampleblock}

  \vspace{0.3cm}

  \begin{alertblock}{解放生产力}
    从重复的信息处理工作中解放出来,专注创造性思考
  \end{alertblock}
\end{frame}

\begin{frame}{成本与价值}
  \begin{columns}
    \begin{column}{0.48\textwidth}
      \textbf{官方订阅}
      \begin{itemize}
        \item 月费:200 美元
        \item 效率提升:5-10 倍
        \item 学习成本低
      \end{itemize}

      \vspace{0.3cm}

      \begin{alertblock}{性价比}
        收益远超成本
      \end{alertblock}
    \end{column}
    \begin{column}{0.48\textwidth}
      \textbf{实际挑战}
      \begin{itemize}
        \item 个人负担较重
        \item 支付手段受限
        \item 可考虑拼单
      \end{itemize}

      \vspace{0.3cm}

      \begin{exampleblock}{建议}
        公司投资 / 团队分摊 / 用替代方案
      \end{exampleblock}
    \end{column}
  \end{columns}
\end{frame}

\begin{frame}{经济替代方案}
  \begin{columns}
    \begin{column}{0.32\textwidth}
      \textbf{GLM-4.6}
      \begin{itemize}
        \item 60 元/季度
        \item Function call 榜单第一
        \item 日常使用足够
      \end{itemize}
    \end{column}
    \begin{column}{0.32\textwidth}
      \textbf{Kimi K2}
      \begin{itemize}
        \item 性能更强
        \item 价格适中
      \end{itemize}
    \end{column}
    \begin{column}{0.32\textwidth}
      \textbf{团队分摊}
      \begin{itemize}
        \item 5 人:40 美元/月
        \item Claude Relay
        \item 需运维能力
      \end{itemize}
    \end{column}
  \end{columns}

  \vspace{0.5cm}

  \begin{exampleblock}{推荐}
    预算有限优先 GLM-4.6,团队可考虑分摊方案
  \end{exampleblock}

  \vspace{0.3cm}

  \small
  Claude Relay Service: \url{https://github.com/Wei-Shaw/claude-relay-service}
\end{frame}

\begin{frame}{AI 能力的本质思考}
  \begin{block}{框架也是能力的一部分}
    Claude Code 证明了 AI 的能力不完全由基础模型决定
  \end{block}

  \vspace{0.3cm}

  \begin{columns}
    \begin{column}{0.48\textwidth}
      \textbf{传统观点}
      \begin{itemize}
        \item AI 能力 = 模型能力
        \item 更大的模型 = 更强的能力
        \item 追求参数规模
      \end{itemize}
    \end{column}
    \begin{column}{0.48\textwidth}
      \textbf{新的认知}
      \begin{itemize}
        \item 工作范式很重要
        \item 框架设计是核心
        \item 多 agent 协作
      \end{itemize}
    \end{column}
  \end{columns}

  \vspace{0.3cm}

  \begin{exampleblock}{实证}
    Haiku 4.5 发布后,看代码功能完全用子 agent 实现
  \end{exampleblock}
\end{frame}

\begin{frame}{多 Agent 架构成为必然}
  \begin{columns}
    \begin{column}{0.48\textwidth}
      \textbf{为什么需要多 Agent?}
      \begin{enumerate}
        \item \textbf{上下文污染问题}
        \begin{itemize}
          \item 一个模型处理所有任务
          \item 不同任务的上下文混杂
          \item 影响推理质量
        \end{itemize}

        \item \textbf{性价比优化}
        \begin{itemize}
          \item 小模型:快速阅读
          \item 大模型:深度思考
          \item 针对性使用资源
        \end{itemize}
      \end{enumerate}

      \vspace{0.3cm}

      \begin{alertblock}{核心}
        上下文隔离 + 成本优化 = 多 Agent 必然趋势
      \end{alertblock}
    \end{column}
    \begin{column}{0.48\textwidth}
      \textbf{Claude Code 的实践}
      \begin{itemize}
        \item Explore Agent:快速搜索
        \item Main Agent:任务编排
        \item Sub Agents:专项处理
        \item 各 Agent 独立上下文
      \end{itemize}

      \vspace{0.3cm}

      \textbf{对基础设施的要求}
      \begin{itemize}
        \item 多模型并发调度
        \item 高效的上下文切换
        \item Agent 间通信机制
        \item 统一的资源管理
        \item \textcolor{blue}{上下文总结与压缩}
      \end{itemize}

      \vspace{0.2cm}

      \begin{exampleblock}{市场影响}
        多 Agent 调用频率 >> 单模型对话
      \end{exampleblock}
    \end{column}
  \end{columns}
\end{frame}

\begin{frame}{对程序员角色的影响}
  \begin{block}{技能广度 vs 技能深度}
    Claude Code 显著增加了程序员的技能广度
  \end{block}

  \vspace{0.3cm}

  \begin{enumerate}
    \item \textbf{分工方式的变化}
    \begin{itemize}
      \item 从技能出发 → 从需求出发
      \item 更接近产品经理的工作方式
    \end{itemize}

    \vspace{0.2cm}

    \item \textbf{FAE 角色越来越重要}
    \begin{itemize}
      \item 现场应用工程师(Field Application Engineer)
      \item 理解客户需求
      \item 转化成 AI 能处理的形式
    \end{itemize}

    \vspace{0.2cm}

    \item \textbf{变革正在发生}
    \begin{itemize}
      \item 从一个人用到许多人用
      \item 当共识形成,下一步变革就会开始
    \end{itemize}
  \end{enumerate}
\end{frame}

\begin{frame}{商业化思考:从 Killer App 到生态}
  \begin{block}{Claude Code 的历史地位}
    第一个真正意义上的 AI Agent Killer App
  \end{block}

  \vspace{0.3cm}

  \begin{columns}
    \begin{column}{0.48\textwidth}
      \textbf{成功的证明}
      \begin{itemize}
        \item 程序员领域:代码开发
        \item 办公领域:文档处理
        \item 真正能干活的 AI
        \item 效率提升 5-10 倍
      \end{itemize}

      \vspace{0.3cm}

      \begin{exampleblock}{核心逻辑}
        如果在这些领域能成为 Killer App,其他领域也可以
      \end{exampleblock}
    \end{column}
    \begin{column}{0.48\textwidth}
      \textbf{必然的扩散}
      \begin{itemize}
        \item 各行各业都有类似需求
        \item 医疗、法律、金融、教育、设计...
        \item 具体形态还不确定
        \item 但趋势已经清晰
      \end{itemize}

      \vspace{0.3cm}

      \begin{alertblock}{商业机会}
        其他领域会出现类似应用,具体形态还在探索中
      \end{alertblock}
    \end{column}
  \end{columns}
\end{frame}

\begin{frame}{市场扩张:Token 需求爆炸性增长}
  \begin{columns}
    \begin{column}{0.48\textwidth}
      \textbf{对话 vs Agent 应用}
      \begin{itemize}
        \item 对话应用:几百到几千 tokens
        \item Agent 应用:几万到几十万 tokens
        \item 持续的上下文处理
        \item 大量文件读写操作
      \end{itemize}

      \vspace{0.3cm}

      \begin{alertblock}{市场扩大}
        每次交互 token 消耗是对话应用的 10-100 倍
      \end{alertblock}

      \vspace{0.3cm}

      \textbf{规模效应}
      \begin{itemize}
        \item 垂直领域扩散 × Token 倍增
        \item AI Infra 市场指数级扩张
      \end{itemize}
    \end{column}
    \begin{column}{0.48\textwidth}
      \textbf{用户行为差异}
      \begin{itemize}
        \item 倾向使用原始 session
        \item 不频繁新开 session
        \item 上下文持续累积
      \end{itemize}

      \vspace{0.3cm}

      \textbf{系统压力}
      \begin{itemize}
        \item 平均上下文长度显著增加
        \item 显存占用大幅提升
        \item KV Cache 管理复杂度增加
      \end{itemize}

      \vspace{0.3cm}

      \begin{exampleblock}{基础设施挑战}
        长上下文场景成为常态,不是例外
      \end{exampleblock}
    \end{column}
  \end{columns}
\end{frame}

\begin{frame}{SLO 目标的根本性变化}
  \begin{columns}
    \begin{column}{0.48\textwidth}
      \textbf{对话应用的 SLO}
      \begin{itemize}
        \item 关注 Prefill 延迟
        \item 关注 Decode TPS
        \item 用户盯着屏幕等输出
        \item 逐字显示很重要
      \end{itemize}

      \vspace{0.3cm}

      \begin{block}{优化目标}
        首 token 时间 + 每秒输出 tokens 数
      \end{block}
    \end{column}
    \begin{column}{0.48\textwidth}
      \textbf{Agent 应用的 SLO}
      \begin{itemize}
        \item 用户不等在屏幕前
        \item 只关心最终结果
        \item Prefill/Decode TPS 概念淡化
        \item 任务完成时间才是核心
      \end{itemize}

      \vspace{0.3cm}

      \begin{alertblock}{优化目标}
        端到端完成时间(End-to-End Latency)
      \end{alertblock}

      \vspace{0.3cm}

      \textbf{架构影响}
      \begin{itemize}
        \item 可以接受更高的 Prefill 延迟
        \item 换取更好的资源利用率
        \item 批处理优化空间更大
      \end{itemize}
    \end{column}
  \end{columns}
\end{frame}

\begin{frame}{Agent 应用:爆发前夜}
  \begin{columns}
    \begin{column}{0.48\textwidth}
      \textbf{Claude Code 的启示}
      \begin{itemize}
        \item 第一个 Killer App
        \item 处于爆发前夜
        \item 主要用户:程序员
      \end{itemize}

      \vspace{0.5cm}

      \textbf{更广阔的需求}
      \begin{itemize}
        \item 其他人也需要智能 agent
        \item 知识库管理
        \item 专业领域应用
        \item 各行各业的效率工具
      \end{itemize}
    \end{column}
    \begin{column}{0.48\textwidth}
      \begin{alertblock}{市场机会}
        Agent 应用爆发 → AI Infra 需求爆发式增长
      \end{alertblock}

      \vspace{0.5cm}

      \textbf{基础设施需求}
      \begin{itemize}
        \item 多模型调度平台
        \item 高性能推理引擎
        \item 成本优化方案
        \item 企业级部署
      \end{itemize}

      \vspace{0.3cm}

      \begin{exampleblock}{趋境科技}
        在 AI Infra 领域占有一席之地
      \end{exampleblock}
    \end{column}
  \end{columns}
\end{frame}

\begin{frame}{总结与展望}
  \begin{columns}
    \begin{column}{0.48\textwidth}
      \textbf{核心价值}
      \begin{itemize}
        \item 全新协作范式
        \item 效率提升 5-10 倍
        \item 适用信息密集型工作
      \end{itemize}

      \vspace{0.5cm}

      \textbf{未来方向}
      \begin{itemize}
        \item 多 agent 系统成熟
        \item 降低技术门槛
        \item 重塑团队结构
      \end{itemize}
    \end{column}
    \begin{column}{0.48\textwidth}
      \begin{alertblock}{变革正在发生}
        从开发、运维到管理,工作方式正在改变
      \end{alertblock}

      \vspace{0.5cm}

      \begin{center}
        \large
        \textbf{愿景}\\
        \vspace{0.3cm}
        解放更多人的生产力
      \end{center}
    \end{column}
  \end{columns}
\end{frame}

%----------------------------------------------------------------------------------------
\section{Q\&A}
%----------------------------------------------------------------------------------------

\begin{frame}[plain]
  \centering
  \Huge \textbf{Q\&A}

  \vspace{1cm}
  \Large
  提问与讨论

  \vspace{1cm}
  \normalsize
  感谢聆听!
\end{frame}

\begin{frame}[plain]
  \centering
  \Huge \textbf{谢谢!}

  \vspace{1cm}
  \normalsize

  \vspace{0.5cm}
  \small
  本演示文稿使用 Claude Code 制作
\end{frame}

\end{document}
