%----------------------------------------------------------------------------------------
%    PACKAGES AND THEMES
%----------------------------------------------------------------------------------------
\documentclass[aspectratio=169,xcolor=dvipsnames]{beamer}
\makeatletter
\def\input@path{{theme/}}
\makeatother
\usetheme{CleanEasy}
\usepackage[utf8]{inputenc}
\usepackage{lmodern}
\usepackage[T1]{fontenc}
\usepackage{fix-cm}
\usepackage{amsmath}
\usepackage{mathtools}
\usepackage{listings}
\usepackage{xcolor}
\usepackage{hyperref}
\usepackage{graphicx}
\usepackage{booktabs}
\usepackage{tikz}
\usetikzlibrary{positioning, shapes, arrows, calc, decorations.pathreplacing, arrows.meta, backgrounds, patterns, overlay-beamer-styles}
\usepackage{etoolbox}
\usepackage{xeCJK}
\setCJKmainfont{STSong}
\setCJKsansfont{STHeiti}
\setCJKmonofont{STFangsong}

%----------------------------------------------------------------------------------------
%    LAYOUT CONFIGURATION
%----------------------------------------------------------------------------------------
\input{configs/configs}
%----------------------------------------------------------------------------------------
%    TITLE PAGE
%----------------------------------------------------------------------------------------


%---------------------------------------------


\title[Claude Code 使用分享]{如何用 Claude Code 提升开发效率}

\author{谢尔文}

\institute{趋动科技}


\vspace{-2cm}\date{2025 年 10 月}
% Define positions for logos on title page
\titlegraphic{
  \begin{tikzpicture}[remember picture, overlay]
    % 
    \node[anchor=south west, xshift=0.5cm, yshift=0cm] at (current page.south west) {
      \includegraphics[height=1.3cm]{logos/CleanEasy-logo1.png}
    };

    % 
    \node[anchor=south west, xshift=2.5cm, yshift=0cm] at (current page.south west) {
      \includegraphics[height=0.9cm]{logos/CleanEasy-logo2.png}
    };
    
    % 
    \node[anchor=north east, xshift=-0.8cm, yshift=-0.3cm] at (current page.north east) {
      \includegraphics[height=1.1cm]{logos/CleanEasy-logo3.png}
    };
    
    %
    \node[anchor=north east, xshift=-2.8cm, yshift=-0.3cm] at (current page.north east) {
      \includegraphics[height=1.4cm]{logos/CleanEasy-logo4.png}
    };
  \end{tikzpicture}
}


%----------------------------------------------------------------------------------------


\begin{document}

\begin{frame}[plain]
  \titlepage
\end{frame}

\begin{frame}[plain]{目录}
  \tableofcontents
\end{frame}

%----------------------------------------------------------------------------------------
\section{开场白}
%----------------------------------------------------------------------------------------

\begin{frame}{我的 Claude Code 使用体验}
  \begin{block}{今年最幸运的一件事}
    用上 Claude Code 是我今年最幸运的事。它给我的震撼不亚于、甚至超过了当初初次使用 ChatGPT 3.5 时的感受。
  \end{block}

  \vspace{0.3cm}

  \begin{columns}
    \begin{column}{0.48\textwidth}
      \textbf{使用时间:}从 8 月至今约 2 个月

      \vspace{0.3cm}

      \textbf{完成的项目:}
      \begin{itemize}
        \item 系统级代码开发
        \item 运维自动化
        \item 各类小工具
        \item 项目管理工作
      \end{itemize}
    \end{column}
    \begin{column}{0.48\textwidth}
      \begin{exampleblock}{全新的协作范式}
        AI 终于能和人站在同一信息起跑线上,充分发挥现代 AI 的真正实力
      \end{exampleblock}

      \vspace{0.3cm}

      \begin{alertblock}{性价比问题}
        虽然每月 200 美元,但是仍然觉得值得
      \end{alertblock}
    \end{column}
  \end{columns}
\end{frame}

%----------------------------------------------------------------------------------------
\section{Claude Code 是什么}
%----------------------------------------------------------------------------------------

\begin{frame}{Claude Code 定位}
  \begin{columns}
    \begin{column}{0.48\textwidth}
      \textbf{核心功能}
      \begin{itemize}
        \item AI 驱动的编程助手
        \item 命令行界面操作
        \item 深度集成文件系统
        \item 自主执行任务
      \end{itemize}
    \end{column}
    \begin{column}{0.48\textwidth}
      \textbf{和其他工具的区别}
      \begin{itemize}
        \item vs IDE:理解更智能
        \item vs Copilot:能完成整个任务
        \item vs ChatGPT:直接操作代码
      \end{itemize}
    \end{column}
  \end{columns}

  \vspace{0.5cm}

  \begin{block}{最大优势}
    基于命令行和文件系统,能拿到海量的上下文信息
  \end{block}
\end{frame}

\begin{frame}{人机协作的本质}
  \begin{block}{效率瓶颈在哪}
    \begin{itemize}
      \item \textbf{人类短板:}打字和说话的速度远比不上 AI
      \item \textbf{人类优势:}强化学习(RL)效率远超 AI
      \item \textbf{AI 优势:}处理海量背景信息
    \end{itemize}
  \end{block}

  \vspace{0.3cm}

  \begin{exampleblock}{最优协作模式}
    AI 负责处理大量背景信息,人类根据实际情况提供新反馈和想法
  \end{exampleblock}

  \vspace{0.3cm}

  \begin{alertblock}{Claude Code 的优势}
    基于命令行和文件系统工作,获取上下文信息的能力极强
  \end{alertblock}
\end{frame}

\begin{frame}{三大核心优势}
  \begin{enumerate}
    \item \textbf{全新的协作范式}
    \begin{itemize}
      \item AI 和人终于站在同一信息起跑线上
      \item 充分发挥现代 AI 的真正实力
    \end{itemize}

    \vspace{0.3cm}

    \item \textbf{真正能干活}
    \begin{itemize}
      \item 不只是提建议,而是直接动手
      \item 自动拆解多步骤任务
    \end{itemize}

    \vspace{0.3cm}

    \item \textbf{工具集成能力强}
    \begin{itemize}
      \item 编辑器、终端、Git、包管理器等一应俱全
      \item 在不同工具间无缝切换
    \end{itemize}
  \end{enumerate}
\end{frame}

%----------------------------------------------------------------------------------------
\section{基础功能}
%----------------------------------------------------------------------------------------

\begin{frame}{读懂代码}
  \begin{columns}
    \begin{column}{0.48\textwidth}
      \textbf{快速理解大项目}
      \begin{itemize}
        \item 自动分析项目结构
        \item 找出关键组件
        \item 理清代码依赖关系
      \end{itemize}
    \end{column}
    \begin{column}{0.48\textwidth}
      \textbf{智能搜索}
      \begin{itemize}
        \item 语义化搜索代码
        \item 跨文件关联分析
        \item 快速定位具体实现
      \end{itemize}
    \end{column}
  \end{columns}
\end{frame}

\begin{frame}{写代码}
  \begin{itemize}
    \item \textbf{智能补全}
    \begin{itemize}
      \item 基于上下文给出精准建议
      \item 理解项目的代码风格
    \end{itemize}

    \vspace{0.3cm}

    \item \textbf{重构和优化}
    \begin{itemize}
      \item 自动发现可优化的地方
      \item 安全地重构代码
    \end{itemize}

    \vspace{0.3cm}

    \item \textbf{自动修 bug}
    \begin{itemize}
      \item 检测和诊断错误
      \item 给出修复方案
    \end{itemize}
  \end{itemize}
\end{frame}

%----------------------------------------------------------------------------------------
\section{高级功能}
%----------------------------------------------------------------------------------------

\begin{frame}{项目管理}
  \begin{columns}
    \begin{column}{0.3\textwidth}
      \textbf{Git 操作}
      \begin{itemize}
        \item 自动提交
        \item 管理分支
        \item 创建 PR
      \end{itemize}
    \end{column}
    \begin{column}{0.3\textwidth}
      \textbf{依赖管理}
      \begin{itemize}
        \item 装包
        \item 版本控制
        \item 解决冲突
      \end{itemize}
    \end{column}
    \begin{column}{0.3\textwidth}
      \textbf{构建测试}
      \begin{itemize}
        \item 自动构建
        \item 跑测试
        \item 配置 CI/CD
      \end{itemize}
    \end{column}
  \end{columns}
\end{frame}

\begin{frame}{任务管理}
  \begin{block}{Todo 列表}
    自动创建和追踪任务,确保不漏掉任何步骤
  \end{block}

  \vspace{0.3cm}

  \begin{exampleblock}{拆解任务}
    把复杂任务拆成一个个小步骤,逐步搞定
  \end{exampleblock}

  \vspace{0.3cm}

  \begin{alertblock}{并行处理}
    识别可以同时做的任务,提高效率
  \end{alertblock}
\end{frame}

\begin{frame}{调试和优化}
  \begin{enumerate}
    \item \textbf{诊断和修复}
    \begin{itemize}
      \item 快速找到错误源头
      \item 理解错误的来龙去脉
      \item 给出修复建议
    \end{itemize}

    \vspace{0.3cm}

    \item \textbf{性能优化}
    \begin{itemize}
      \item 找出性能瓶颈
      \item 提供优化方案
      \item 实施并验证改进效果
    \end{itemize}

    \vspace{0.3cm}

    \item \textbf{代码质量}
    \begin{itemize}
      \item 检查代码规范
      \item 建议最佳实践
      \item 审查安全性
    \end{itemize}
  \end{enumerate}
\end{frame}

%----------------------------------------------------------------------------------------
\section{真实案例}
%----------------------------------------------------------------------------------------

\begin{frame}{案例 1:快速看懂代码 - AWQ 量化格式研究}
  \begin{block}{任务}
    在 sglang 代码库里找到 AWQ 量化的 swizzle 格式是怎么排列的
  \end{block}

  \begin{columns}
    \begin{column}{0.48\textwidth}
      \textbf{技术背景}
      \begin{itemize}
        \item AWQ:4bit 量化格式
        \item 不按行主序/列主序排
        \item 用 Swizzle 顺序排列权重
        \item 论文里说得不清楚
      \end{itemize}
    \end{column}
    \begin{column}{0.48\textwidth}
      \textbf{效率对比}
      \begin{itemize}
        \item \textcolor{green}{用 Claude Code:几分钟}
        \item \textcolor{red}{传统方式:几小时}
        \item 对代码库完全不熟的情况下
        \item 需要深入理解底层实现
      \end{itemize}
    \end{column}
  \end{columns}

  \vspace{0.3cm}

  \begin{exampleblock}{核心价值}
    快速看懂陌生代码库,精准定位关键实现
  \end{exampleblock}
\end{frame}

\begin{frame}{案例 2:系统级开发 - KTransformers 支持 AWQ}
  \begin{block}{任务}
    实现 group 量化(group size = 64)
  \end{block}

  \begin{columns}
    \begin{column}{0.48\textwidth}
      \textbf{技术难点}
      \begin{itemize}
        \item K 维度需要分块
        \item 高维矩阵切分和 pack
        \item Per-channel 改 Per-group
        \item AVX 指令优化
      \end{itemize}
    \end{column}
    \begin{column}{0.48\textwidth}
      \textbf{协作流程}
      \begin{enumerate}
        \item AI 先写测试用例
        \item 参考原方案改代码
        \item 生成基本框架
        \item 人工调 AVX 指令
        \item 成功跑通
      \end{enumerate}
    \end{column}
  \end{columns}

  \vspace{0.3cm}

  \begin{alertblock}{效率提升}
    项目周期从 \textcolor{red}{2 周} 缩短到 \textcolor{green}{2 天}
  \end{alertblock}
\end{frame}

\begin{frame}{案例 3:运维 - GPU 监控中心部署}
  \begin{block}{背景}
    多个 GPU 集群要管理,还得解决 GPU 抢占和资源分配的问题
  \end{block}

  \begin{columns}
    \begin{column}{0.48\textwidth}
      \textbf{技术方案}
      \begin{itemize}
        \item Node Exporter(系统监控)
        \item NVIDIA DCGM Exporter(GPU 监控)
        \item Grafana(可视化)
        \item 自己写的 Dashboard
      \end{itemize}
    \end{column}
    \begin{column}{0.48\textwidth}
      \textbf{做了什么}
      \begin{itemize}
        \item 自动化部署脚本
        \item Docker 容器化
        \item 批量远程部署
        \item 聚合监控数据
        \item 多机器 GPU 看板
      \end{itemize}
    \end{column}
  \end{columns}

  \vspace{0.3cm}

  \begin{exampleblock}{完全 Video Coding}
    \textcolor{green}{4 小时}搞定 vs 传统方式 \textcolor{red}{2-3 天}
  \end{exampleblock}

  \small{主要时间花在解决 Docker 源和依赖问题上}
\end{frame}

\begin{frame}{案例 4:文档工程 - 立项文档自动化}
  \begin{block}{场景}
    KTransformers 和华为合作的立项文档
  \end{block}

  \begin{columns}
    \begin{column}{0.48\textwidth}
      \textbf{传统方式的痛点}
      \begin{itemize}
        \item 疯狂复制粘贴
        \item 反复调格式
        \item 信息整合麻烦
        \item 版本同步困难
      \end{itemize}
    \end{column}
    \begin{column}{0.48\textwidth}
      \textbf{用 Claude Code 的流程}
      \begin{enumerate}
        \item 准备背景资料和模板
        \item 转成 Markdown
        \item AI 生成初稿
        \item 在 MD 里调整
        \item 复制到 Word 里
      \end{enumerate}
    \end{column}
  \end{columns}

  \vspace{0.3cm}

  \begin{exampleblock}{效果}
    写文档变得又快又轻松,专注内容而不是格式
  \end{exampleblock}
\end{frame}

\begin{frame}{案例 5:管理自动化 - 项目进度管理系统}
  \begin{block}{需求}
    多个项目要对接同步,还得向上汇报、追踪进度
  \end{block}

  \begin{columns}
    \begin{column}{0.48\textwidth}
      \textbf{解决方案}
      \begin{itemize}
        \item 自动化进度更新
        \item 多项目信息聚合
        \item 自动生成日报
        \item 自动生成周报
        \item 数据可视化
      \end{itemize}
    \end{column}
    \begin{column}{0.48\textwidth}
      \textbf{带来的价值}
      \begin{itemize}
        \item 减少重复劳动
        \item 汇报更准确
        \item 节省管理时间
        \item 方便追踪进度
        \item 管理效率大幅提升
      \end{itemize}
    \end{column}
  \end{columns}

  \vspace{0.3cm}

  \begin{alertblock}{关键洞察}
    把管理工作系统化、自动化,腾出时间来思考
  \end{alertblock}
\end{frame}

%----------------------------------------------------------------------------------------
\section{最佳实践}
%----------------------------------------------------------------------------------------

\begin{frame}{怎么问问题}
  \begin{enumerate}
    \item \textbf{清楚地描述}
    \begin{itemize}
      \item 说清楚想要什么结果
      \item 描述现在是什么状态
      \item 贴上错误信息
    \end{itemize}

    \vspace{0.3cm}

    \item \textbf{给足上下文}
    \begin{itemize}
      \item 指出相关文件
      \item 说明用的什么技术栈
      \item 讲清楚有什么限制
    \end{itemize}

    \vspace{0.3cm}

    \item \textbf{一步步来}
    \begin{itemize}
      \item 把复杂任务拆开
      \item 每步都验证结果
      \item 及时给反馈
    \end{itemize}
  \end{enumerate}
\end{frame}

\begin{frame}{优化工作流}
  \begin{columns}
    \begin{column}{0.3\textwidth}
      \textbf{工具集成}
      \begin{itemize}
        \item Git
        \item IDE
        \item 终端
        \item CI/CD
      \end{itemize}
    \end{column}
    \begin{column}{0.3\textwidth}
      \textbf{团队协作}
      \begin{itemize}
        \item 代码规范
        \item 文档同步
        \item 知识共享
        \item 最佳实践
      \end{itemize}
    \end{column}
    \begin{column}{0.3\textwidth}
      \textbf{提效技巧}
      \begin{itemize}
        \item 快捷命令
        \item 用模板
        \item 批量操作
        \item 自动化脚本
      \end{itemize}
    \end{column}
  \end{columns}
\end{frame}

%----------------------------------------------------------------------------------------
\section{挑战与解决方案}
%----------------------------------------------------------------------------------------

\begin{frame}{上下文限制问题}
  \begin{block}{现状}
    Claude Code 的上下文一般在几百 K 级别,开发大项目时可能一个特性就用完
  \end{block}

  \vspace{0.3cm}

  \begin{columns}
    \begin{column}{0.48\textwidth}
      \textbf{解决方案}
      \begin{itemize}
        \item 让 AI 写文档记录状态
        \item 定期重构项目结构
        \item 模块化开发
        \item 分阶段处理任务
      \end{itemize}
    \end{column}
    \begin{column}{0.48\textwidth}
      \textbf{为什么不能简单扩展}
      \begin{itemize}
        \item 长上下文成本太高
        \item 信息检索效率降低
        \item 多 agent 是更优解
      \end{itemize}
    \end{column}
  \end{columns}

  \vspace{0.3cm}

  \begin{alertblock}{未来趋势}
    多 agent 系统会成为主流(Haiku 4.5 已采用)
  \end{alertblock}
\end{frame}

%----------------------------------------------------------------------------------------
\section{核心思考}
%----------------------------------------------------------------------------------------

\begin{frame}{Claude Code 的本质}
  \begin{block}{信息处理加速器}
    任何需要快速处理和输出信息的工作都可以交给它
  \end{block}

  \vspace{0.3cm}

  \begin{exampleblock}{优化流程的思维}
    人负责思考整个流程哪里可以优化
  \end{exampleblock}

  \vspace{0.3cm}

  \begin{alertblock}{解放生产力}
    从重复的信息处理工作中解放出来,专注创造性思考
  \end{alertblock}
\end{frame}

\begin{frame}{性价比分析}
  \begin{columns}
    \begin{column}{0.48\textwidth}
      \textbf{成本}
      \begin{itemize}
        \item 月费:200 美元
        \item 学习成本:很低
        \item 适应时间:很短
      \end{itemize}

      \vspace{0.5cm}

      \begin{block}{总投入}
        每月 200 美元 + 一点学习时间
      \end{block}
    \end{column}
    \begin{column}{0.48\textwidth}
      \textbf{收益}
      \begin{itemize}
        \item 效率提升:5-10 倍
        \item 能力扩展:显著
        \item 项目交付:更快
        \item 收入潜力:大幅提升
      \end{itemize}

      \vspace{0.5cm}

      \begin{alertblock}{性价比}
        超值,简直是捡了大便宜
      \end{alertblock}
    \end{column}
  \end{columns}
\end{frame}

\begin{frame}{价格问题与解决方案}
  \begin{alertblock}{价格确实是个问题}
    每月 200 美元对个人用户确实是不小的负担
  \end{alertblock}

  \vspace{0.3cm}

  \begin{columns}
    \begin{column}{0.48\textwidth}
      \textbf{国内使用的实际问题}
      \begin{itemize}
        \item 缺乏合适的支付手段
        \item 需要跨境支付机构
        \item 可以考虑和同学拼单
      \end{itemize}
    \end{column}
    \begin{column}{0.48\textwidth}
      \textbf{对不同用户的价值}
      \begin{itemize}
        \item 个人:需权衡投入产出
        \item 公司:效率提升值得投资
        \item 学生:可以拼单降低成本
      \end{itemize}
    \end{column}
  \end{columns}
\end{frame}

\begin{frame}{更经济的替代方案}
  \begin{block}{Claude Code 支持多种模型}
    除了 Anthropic 官方模型,还可接入其他模型
  \end{block}

  \vspace{0.3cm}

  \begin{columns}
    \begin{column}{0.48\textwidth}
      \textbf{GLM-4.6}
      \begin{itemize}
        \item 开源模型
        \item 价格:60 元/季度
        \item 在 BFCL 等 function call 榜单第一
        \item 日常使用基本没问题
      \end{itemize}
    \end{column}
    \begin{column}{0.48\textwidth}
      \textbf{Kimi K2}
      \begin{itemize}
        \item 模型尺寸更大
        \item 性能更强
        \item 价格介于两者之间
      \end{itemize}
    \end{column}
  \end{columns}

  \vspace{0.3cm}

  \begin{exampleblock}{推荐}
    如果觉得 Claude 价格太贵,可以考虑用 GLM-4.6
  \end{exampleblock}
\end{frame}

\begin{frame}{AI 能力的本质思考}
  \begin{block}{框架也是能力的一部分}
    Claude Code 证明了 AI 的能力不完全由基础模型决定
  \end{block}

  \vspace{0.3cm}

  \begin{columns}
    \begin{column}{0.48\textwidth}
      \textbf{传统观点}
      \begin{itemize}
        \item AI 能力 = 模型能力
        \item 更大的模型 = 更强的能力
        \item 追求参数规模
      \end{itemize}
    \end{column}
    \begin{column}{0.48\textwidth}
      \textbf{新的认知}
      \begin{itemize}
        \item 工作范式很重要
        \item 框架设计是核心
        \item 多 agent 协作
      \end{itemize}
    \end{column}
  \end{columns}

  \vspace{0.3cm}

  \begin{exampleblock}{实证}
    Haiku 4.5 发布后,看代码功能完全用子 agent 实现
  \end{exampleblock}
\end{frame}

\begin{frame}{对程序员角色的影响}
  \begin{block}{技能广度 vs 技能深度}
    Claude Code 显著增加了程序员的技能广度
  \end{block}

  \vspace{0.3cm}

  \begin{enumerate}
    \item \textbf{分工方式的变化}
    \begin{itemize}
      \item 从技能出发 → 从需求出发
      \item 更接近产品经理的工作方式
    \end{itemize}

    \vspace{0.2cm}

    \item \textbf{FAE 角色越来越重要}
    \begin{itemize}
      \item 现场应用工程师(Field Application Engineer)
      \item 理解客户需求
      \item 转化成 AI 能处理的形式
    \end{itemize}

    \vspace{0.2cm}

    \item \textbf{变革正在发生}
    \begin{itemize}
      \item 从一个人用到许多人用
      \item 当共识形成,下一步变革就会开始
    \end{itemize}
  \end{enumerate}
\end{frame}

\begin{frame}{未来展望}
  \begin{columns}
    \begin{column}{0.48\textwidth}
      \textbf{Claude Code 的进化}
      \begin{itemize}
        \item 理解能力更强
        \item 集成更多工具
        \item 协作体验更好
        \item 任务执行更智能
        \item 多 agent 系统成熟
      \end{itemize}
    \end{column}
    \begin{column}{0.48\textwidth}
      \textbf{AI 协作范式的影响}
      \begin{itemize}
        \item 改变开发方式
        \item 提升生产力水平
        \item 降低技术门槛
        \item 加速创新速度
        \item 重塑团队结构
      \end{itemize}
    \end{column}
  \end{columns}

  \vspace{0.5cm}

  \begin{center}
    \large
    \textbf{愿景:解放更多人的生产力}
  \end{center}
\end{frame}

\begin{frame}{核心观点}
  \begin{enumerate}
    \item Claude Code 带来了全新的协作范式,AI 和人终于站在同一信息起跑线上

    \vspace{0.3cm}

    \item 最优协作模式:AI 处理背景信息 + 人类提供反馈和想法

    \vspace{0.3cm}

    \item 从多个真实案例看:效率提升 5-10 倍是常态

    \vspace{0.3cm}

    \item 月费 200 美元的投资带来巨大回报

    \vspace{0.3cm}

    \item 任何需要快速处理和输出信息的工作都可以交给 Claude Code
  \end{enumerate}

  \vspace{0.3cm}

  \begin{center}
    \Large
    感谢 Claude Code,让我有能力实现我的各种想法,希望未来它会带来进一步的变革。
  \end{center}
\end{frame}

%----------------------------------------------------------------------------------------
\section{Q\&A}
%----------------------------------------------------------------------------------------

\begin{frame}[plain]
  \centering
  \Huge \textbf{Q\&A}

  \vspace{1cm}
  \Large
  提问与讨论

  \vspace{1cm}
  \normalsize
  感谢聆听!
\end{frame}

\begin{frame}[plain]
  \centering
  \Huge \textbf{谢谢!}

  \vspace{1cm}
  \normalsize

  \vspace{0.5cm}
  \small
  本演示文稿使用 Claude Code 制作
\end{frame}

\end{document}
